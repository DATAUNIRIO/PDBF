\documentclass[11pt]{article}
\usepackage[utf8]{inputenc}
\usepackage{pdbf}
\usepackage{wrapfig}
\usepackage[hidelinks]{hyperref}
\usepackage{listings}
\usepackage[a4paper,vmargin={20mm,40mm},hmargin={20mm,20mm}]{geometry}

\pdfobjcompresslevel=2

\title{\href{https://github.com/uds-datalab/PDBF}{PDBF} Documentation}
\author{
 Patrick Bender \hspace{1cm} Lukas Lange \hspace{1cm} Shoaib Malik\\
  \texttt{\{s9pabend, s9lslang, s9mumali\}@stud.uni-saarland.de}
}
\date{}

\hypersetup{
  colorlinks=true,
  linkcolor=black,
  urlcolor=blue!70!black
}

\makeatletter
\def\@maketitle{%
  \vspace*{-\topskip}%
  \begingroup\centering%
  \let \footnote \thanks
  \hrule height \z@%
    {\LARGE \@title \par}%
    \vskip 1.5em 
    {\large
      \lineskip .5em 
      \begin{tabular}[t]{c}%
        \@author
      \end{tabular}\par}%
    \vskip 1em 
    {\large \@date}%
  \par\endgroup%
  \vskip 1.5em%
}
\makeatother

\urlstyle{rm}

\def\a{5cm}
\def\b{11.2cm}

\def\option#1#2#3#4{%
\noindent \begin{tabular}{|p{\a}|p{\b}|}
\hline
\textbf{#1} & \\
\hline
Desc.: & #2 \\
\hline
Argument: & #3\\
\hline
Default:& #4\\
\hline
\end{tabular} \\[4pt]%
}

\def\pdbfQuality{2.0}

\dbSQLText{CREATE TABLE test(a int, b int); INSERT INTO test VALUES(1, 10); INSERT INTO test VALUES(2, 100); INSERT INTO test VALUES(3, 120); INSERT INTO test VALUES(4, 100);}

\jsonFile{examples/json/nested/cake.json}{cake}

\begin{document}
\maketitle
\noindent \textbf{Website:} \url{https://github.com/uds-datalab/PDBF}\\
\noindent \textbf{\input{VERSION.md}}

\tableofcontents
\newpage

\section{Bugs, Suggestions, Feature requests}
If you encounter bugs, have suggestions or have a feature request, then please go to the issue page open a new issue if necessary and explain your concern. You can also write us an email (ichbinkeinreh at t-online.de or jens.dittrich at cs.uni-saarland.de).

\section{Abstract}
PDBF documents are a hybrid format. They are a valid PDF and a valid HTML page at the same time. You can now optionally add an Open Virtual Appliance (OVA) file with a complete operating system to the PDBF document. Yes, this means that the resulting file is a valid PDF, HTML, and OVA file at the same time. If you change the file extension to PDF and open it with an PDF viewer, you can see the static part of the document. If you change the file extension to HTML and open it with a Browser (currently Chrome/Firefox/Safari/IE 10 supported), you can see the dynamic part of the document. And if an ova file is attached you can also change the file extension to OVA and install and run the attached operating system.\\
The difference between the PDF and the HTML version is that the PDF version contains static version of all PDBF elements, whereas the HTML version is dynamic. For example you can zoom into graphs, temporarly remove dataseries from the graph, inspect and change the underling query of the PDBF element and see the result of the change directly in the browser.\\
This approach works completely offline. No internet connection is required, neither at compile time, nor at viewing time.

\noindent PDBF files are created from LaTeX source code and a relational database. The raw data can either be a SQL statement string, a file with SQL statements, or contained in a database (currently PostgreSQL/MySQL/MariaDB supported). In the LaTeX code you can then specify how the PDBF element (currently charts/pivot tables/multiplot charts/sql statements/dataTexts/dataTables are supported) is created from the raw data with options and an SQL query. Read more in the \href{http://uds-datalab.github.io/PDBF/}{documentation}, which is itself is a PDBF document.\\

\noindent PDBF toolkit is written in Java and LaTeX and can be used to compile documents on Windows, Mac, and Linux. PDBF documents are also platform independent and run on any desktop OS (Windows, Linux, Mac) with a browser/PDF viewer.\\

\noindent A \href{https://infosys.uni-saarland.de/publications/p1972-dittrich.html}{demo paper} of our tool appeared at \href{http://www.vldb.org/2015/}{VLDB 2015}. 

\section{License}
\noindent This toolkit is licensed unter the MIT License. See LICENSE.md file.\\

\newpage

\section{Getting started}
\subsection{Normal usage}
\begin{itemize}
\item Make sure you have a Java Runtime (version $>=$ 1.7) and a LaTeX distribution installed
\item \href{https://github.com/uds-datalab/PDBF/archive/gh-pages.zip}{Download the latest version}
\item Extract zip and change workingdir to extracted folder
\item Adjust config.cfg
\item Try to compile minimal.tex file with this command: \texttt{java -jar pdbf.jar minimal.tex}
\item Open minimal.html, this is the final output of the compilation process, if you rename it to ".pdf" it is also a valid pdf-document
\end{itemize}
\subsection{Attach a Open Virtual Appliance (OVA) file}
\begin{itemize}
\item Optionally you can attach the included vldb-Invaders.ova (Space invaders clone) or \href{https://github.com/uds-datalab/PDBF/releases/download/1.0.1/dsl.ova}{download the dsl.ova} (Damn small linux) Open Virtual Appliance (OVA) file and attach it to the compiled PDBF file with this command: \texttt{java -jar pdbf.jar --vm minimal.html vldb-Invaders.ova}.
\item Open minimal.ova, this is the final output with the attached ova file. Its still a valid pdf and html file at the same time.
\end{itemize}
\subsection{Attach a tar archive}
\textbf{Note:} compressed tar.gz archives are not supported. You can only wrap your tar.gz file in another tar and then attach it.
\begin{itemize}
\item Optionally you can attach a tar archive file to the compiled PDBF file with this command: \texttt{java -jar pdbf.jar --tar minimal.html TAR\_file.tar}
\item Open minimal.tar, this is the final output with the attached tar file. Its still a valid pdf and html file at the same time.
\end{itemize}
\subsection{Learn how to use the PDBF framework}
\begin{itemize}
\item You can play around with the minimal.tex file. It contains a small example on how to specify PDBF elements in LaTeX
\item For further information take a look at the \href{http://uds-datalab.github.io/PDBF/}{documentation}
\end{itemize}

\section{Features}
\subsection{Automatic generation of Charts, Multiplot Charts, Pivot tables}
With PDBF you don't need to manually generate these kinds of elements. The PDBF compiler automatically generates a static version for the pdf and the dynamic version for the html part of the PDBF document. This also means that your document is always up to date! If you change something in the underlying data that generate your PDBF document and then recompile the document, then the data in the document is up to date. No need to manually update externally generated charts or pivot tables.

\subsection{Generate your document directly from the results of your experiment}
The idea of PDBF is to store the results of the experiment directly in the document and to make it more transparent how this chart, pivot table, etc. was generated from the result data. Therefore we currently support CSV files, SQL files, and SQL servers as data sources and use SQL as description language for the transformation of the raw result data to the final representation in the document.

\subsection{Enable the comparison of your results to other users}
To make your own results even more transparent, you can let other users input their own result files. If there is a difference, the PDBF will recognize it, and will automaticly create a highlighted view of changes. However right now, this option is only avaible for JSON-files as datasource. 

\subsection{Compile LaTeX to single HTML file}
You can also use the PDBF compiler to compile your LaTeX files to a single HTML file.
To do so just run the compiler as on any other document (you dont need to include the pdbf package in your tex file):\\[3pt]
\texttt{java -jar pdbf.jar sometexfile.tex}\\[3pt]
The resulting HTML file is saved in the same folder with the same name but with ".html" filename extension.

\section{Build Instructions}
\begin{itemize}
\item Run "mvn package" if you only want to compile pdbf.jar
\item Run "mvn verify" if you want to compile pdbf.jar and run integration tests.
\end{itemize}

\section{Thanks to}
phantomJS (\url{https://github.com/ariya/phantomjs})\\
Apache Commons (\url{http://commons.apache.org/})\\
Apache PDFBox (\url{https://pdfbox.apache.org/})\\
jTar (\url{https://github.com/kamranzafar/jtar})\\
google-gson (\url{https://github.com/google/gson})\\
postgreSQL JDBC Driver (\url{https://jdbc.postgresql.org/})\\
MariaDB JDBC Driver (\url{https://mariadb.com/kb/en/mariadb/about-the-mariadb-java-client/})\\
AlaSQL (\url{https://github.com/agershun/alasql})\\
C3 (\url{https://github.com/masayuki0812/c3})\\
D3 (\url{https://github.com/mbostock/d3})\\
Codemirror (\url{https://github.com/codemirror/codemirror})\\
google-diff-match-patch (\url{https://code.google.com/p/google-diff-match-patch/})\\
jsondiffpatch (\url{https://github.com/benjamine/jsondiffpatch}) \\
explorercanvas (\url{https://code.google.com/p/explorercanvas/})\\
DataTables (\url{https://github.com/DataTables/DataTables})\\
jQuery (\url{https://github.com/jquery/jquery})\\
jQuery-UI (\url{https://github.com/jquery/jquery-ui})\\
jStat (\url{https://github.com/jstat/jstat})\\
pivottable (\url{https://github.com/nicolaskruchten/pivottable})\\
PDF.js (\url{https://github.com/mozilla/pdf.js})\\
google-closure-compiler (\url{https://github.com/google/closure-compiler})\\
lz-string (\url{https://github.com/pieroxy/lz-string})\\
YUI Compressor (\url{https://github.com/yui/yuicompressor})\\

\newpage

\section{How to use the PDBF compiler}
-- The standart mode which compiles a tex file into the janiform PDBF format:\\
\verb|java -jar PDBF.jar LaTeX_file| \\[4pt]
\textbf{Note:} The compiled document can be found in the same folder as the tex file and has the same name as the tex file but ".html" as filename extension.\\[8pt]
-- The VM mode which attaches a Open Virtual Appliance (OVA) file to an existing PDBF document:\\
\verb|java -jar PDBF.jar --vm PDBF_File.html VM_File.ova|\\[4pt]
\textbf{Note:} The compiled document can be found in the same folder as the html file and has the same name as the html file but ".ova" as filename extension.\\[8pt]
-- The TAR mode which attaches a tar archive file to an existing PDBF document:\\
\verb|java -jar PDBF.jar --tar PDBF_File.html TAR_File.tar|\\[4pt]
\textbf{Note:} The compiled document can be found in the same folder as the html file and has the same name as the html file but ".tar" as filename extension.\\[2pt]
\textbf{Note:} compressed tar.gz archives are not supported. You can only wrap your tar.gz file in another tar and then attach it.

\section{Requirements}
The following \LaTeX~packages are required for PDBF documents:
\begin{itemize}
\item \verb|zref|
\item \verb|xcolor|
\item \verb|graphicx|
\item \verb|xstring|
\item \verb|xparse|
\item \verb|geometry|
\item \verb|array|
\end{itemize}
\noindent Furthermore it is needed that your \LaTeX~document uses the geometry package to specify the page size.\\

\noindent \textbf{Warning:} There are problems with \textbf{inputenc} and \textbf{pgfpages} package. If you really have to use the inputenc package with UTF8 option then you cant use non-ASCII characters inside queries. The pgfpages package is currently incompatible to this package. 

\section{SQL specifics and special functions}
PDBF uses AlaSQL as database engine. For all available features of the database visit \url{https://github.com/agershun/alasql}. Some functionality might not yet be available because we currently don't use the latest version.\\[4pt]
\textbf{Note:} Mostly sql syntax is standart sql, but for example attributes with spaces in their name have to be sourrounded by square brackets. E.g. SELECT attribute AS [attribute with spaces] FROM test;\\

\noindent We extended AlaSQL with the following statistical sql functions:\\[5pt]
\verb|GRUBBS_FILTER(arr, alpha, max)|: \\
\indent Desc.: Filters an array of values with a two-sided \href{http://en.wikipedia.org/wiki/Grubbs'_test_for_outliers}{grubbs test}. The grubbs test is performed on the whole array. If a significant outlier is detected (Z > ZCrit) and the relative margin of error is not too high (margin\_of\_error / avg > 0.025) then the outlier is removed. This procedure is repeated until one of these two conditions fails.\\
\indent Returns the filtered array.\\[3pt]
\indent \option
{arr}
{Array on which the operation is performed.}
{JavaScript array}
{No default value. Value must always be set!}
\indent \option
{alpha}
{significance level}
{Number $>$ 0 AND Number $<$ 1}
{\texttt{0.05}}\\[3pt]
%
\verb|MEAN(arr)|: \\
\indent Desc.: Calculates the \href{http://en.wikipedia.org/wiki/Arithmetic_mean}{arithmetic mean} of an array of values.\\
\indent Returns the arithmetic mean of the array.\\[3pt]
\indent \option
{arr}
{Array on which the operation is performed.}
{JavaScript array}
{No default value. Value must always be set!}\\[3pt]
%
\verb|STDDEV_SAMP(arr)|: \\
\indent Desc.: Calculates the \href{http://en.wikipedia.org/wiki/Standard_deviation}{ sample standard deviation} of an array of values.\\
\indent Returns the sample standard deviation of the array.\\[3pt]
\indent \option
{arr}
{Array on which the operation is performed.}
{JavaScript array}
{No default value. Value must always be set!}\\[3pt]
%
\verb|MARGIN_OF_ERROR(arr, alpha)|: \\
\indent Desc.: Calculates the \href{http://en.wikipedia.org/wiki/Margin_of_error}{margin of error} of an array of values.\\
\indent Returns the margin of error of the array.\\[3pt]
\indent \option
{arr}
{Array on which the operation is performed.}
{JavaScript array}
{No default value. Value must always be set!}
\indent \option
{alpha}
{significance level}
{Number $>$ 0 AND Number $<$ 1}
{\texttt{0.05}}\\[3pt]
%
\verb|CONF_INT(arr, alpha)|: \\
\indent Desc.: Calculates the \href{http://en.wikipedia.org/wiki/Confidence_interval}{confidence interval} of an array.\\
\indent Returns the confidence interval of the array.\\[3pt]
\indent \option
{arr}
{Array on which the operation is performed.}
{JavaScript array}
{No default value. Value must always be set!}
\indent \option
{alpha}
{significance level}
{Number $>$ 0 AND Number $<$ 1}
{\texttt{0.05}}\\[3pt]
%
\verb|T_TEST(arr1, arr2, alpha)|: \\
\indent Desc.: Performs a \href{http://en.wikipedia.org/wiki/Student's_t-test}{Student's t-test} on two arrays.\\
\indent Returns true if they are indistinguishable, and returns false otherwise.\\[3pt]
\indent \option
{arr1, arr2}
{Arrays on which the operation is performed.}
{JavaScript array}
{No default value. Value must always be set!}
\indent \option
{alpha}
{significance level}
{Number $>$ 0 AND Number $<$ 1}
{\texttt{0.05}}\\[3pt]
%
\verb|WELCH_TEST(arr1, arr2, alpha)|:\\
\indent Desc.: Performs a \href{http://en.wikipedia.org/wiki/Welch's_t_test}{Welch's t-test} on two arrays.\\
\indent Returns true if they are indistinguishable, and returns false otherwise.\\[3pt]
\indent \option
{arr1, arr2}
{Arrays on which the operation is performed.}
{JavaScript array}
{No default value. Value must always be set!}
\indent \option
{alpha}
{significance level}
{Number $>$ 0 AND Number $<$ 1}
{\texttt{0.05}}\\[3pt]
%
\newpage
\section{Data sources}

\subsection{SQL}
\noindent Macros: \\[3pt]
\verb|\dbSQLText{sqlQueryString}| \\
\verb|\dbSQLFile{fileWithSQLQueries}| \\
\textbf{Note:} Relative filepaths are relative to the tex file.\\
\verb|\dbSQLJDBC{jdbcConnectionURL}{user}{password}{commaSeperatedListOfTableNames}| \\
\textbf{Note:} jdbcConnectionURL consist of jdbc followed by the name of the dbms (only postgresql and mysql are currently supported) followed by the url of the dbms followed by the database name (e.g. jdbc:postgresql://localhost:5432/postgres).\\
\textbf{Note:} jdbcConnectionURL, user, and password are not stored in the output documents.\\[4pt]

\subsection{JSON}
\noindent Macros: \\[3pt]
\verb|\jsonFile{path/to/file}{filename} | \\
\verb|\jsonDirectory{path/to/directory}| \\
\textbf{Note:} Relative filepaths are relative to the tex file.\\
\textbf{Note:} The \verb|\jsonDirectory|-command will search the whole directory (it will not scan deeper file structures), and chooses the filename according to existing filename. \\
\textbf{Note:} You can insert as many JSON-commands as you like. \\[3pt]

\subsection{CSV}
\textbf{Note:} By default the first line of the CSV file is used as attribute names for the table. You can use the header option to manually specify attribute names.\\[3pt]
\verb|\dbCSVFile[options]{fileWithCSVData}{tableName}| \\[8pt]
\textbf{Note:} Relative filepaths are relative to the tex file.\\[4pt]
\noindent Options:\\[3pt]
%
\option
{headers}
{Defines the names of columns for the CSV file. If not set the first line of the CSV file is used as header.}
{Javascript-Array of Strings}
{\texttt{[]}}

\option
{quote}
{Defines the quote character for this CSV file.}
{String}
{\texttt{"}}

\option
{seperator}
{Defines the seperator character for this CSV file.}
{String}
{\texttt{,}}

\newpage
\section{PDBF elements}
\textbf{Note:} All PDBF elements use the font size, font family, and font style that was active at the point of their definition. You can for example surround all commands with \verb|\textbf{...}|, \verb|\emph{...}| or any other font command.\\[3pt]
\subsection{Chart}
\begin{figure}[h!]%
\hspace{-218pt}Examples:\\
    \centering
    \begin{minipage}{.48\textwidth}
    \chart[width=\textwidth, height=0.6\textwidth, chartType=line]{SELECT * FROM test;}
    \caption{Line Chart}
    \end{minipage}
    \hspace{11pt}
    \begin{minipage}{.48\textwidth}
    \chart[width=\textwidth, height=0.6\textwidth, chartType=bar]{SELECT * FROM test;}
    \caption{Bar Chart}
    \end{minipage} \\[8pt]
    \begin{minipage}{1.0\textwidth}
    \chart[width=\textwidth, height=0.36\textwidth, chartType=compareToBest]{SELECT b FROM test;}
    \caption{compareToBest Chart}
    \end{minipage}
\end{figure}

\noindent Macros: \\[3pt]
\verb|\chart[options][queryForOverlay]{queryForPage}| \\
\textbf{Note:} if \verb|queryForOverlay| is omitted the \verb|queryForPage| is used for the overlay.\\
\textbf{Note:} The first column of the query result is used for the x-Axis, all other columns are used for y-Axis.\\[8pt] 
\noindent Options:\\[3pt]
%
\option
{name}
{Sets the name of this element. This is only useful for linking Elements. See DataText.}
{String}
{\texttt{undefined}}

\option
{width}
{Sets the width of the chart.}
{\LaTeX~length}
{No default value. Value must always be set!}

\option
{height}
{Sets the height of the chart.}
{\LaTeX~length}
{No default value. Value must always be set!}

\option
{quality}
{Sets the quality for the image version of the chart in the pdf. 1.0 corresponds roughly to 240 pixels per inch. Can also be redefined globally (\texttt{pdbfQuality}).}
{Number $>$ 0}
{\texttt{1.0}}

\option
{chartType}
{Sets the type of the chart.}
{\texttt{line} or \texttt{bar} or \texttt{compareToBest}}
{\texttt{line}}

\noindent\textbf{Note:} A chart or multiplot chart with option "chartType" set to "compareToBest" expects either an array of numbers or an array of an array of numbers as input. It plots the slowdown compared to the best value in the array. In the second case the inner array of numbers contains values for repetitions of the same experiment. \\[6pt]

\option
{xunit}
{Sets the a label for the x-axis.}
{String}
{\texttt{""} (which means hide)}

\option
{yunit}
{Sets the a label for the y-axis.}
{String}
{\texttt{""} (which means hide)}

\option
{options}
{Sets options that are directly passed to the c3 chart library (\href{http://c3js.org/reference.html}{$\rightarrow$ Documentation}). You need to wrap the JSON-String with \texttt{\{\}} if you want to use \texttt{[} or \texttt{]} or \texttt{,}.}
{JSON-String}
{\texttt{\{\}} (which means empty object)}

\option
{includeZero}
{Set the minimum of the range of the y-axis to zero.}
{\texttt{true} or \texttt{false}}
{\texttt{false}}

\option
{drawPoints}
{Whether to show each point in line.}
{\texttt{true} or \texttt{false}}
{\texttt{false}}

\option
{fillGraph}
{Whether to fill the area below the graph.}
{\texttt{true} or \texttt{false}}
{\texttt{false}}

\option
{showRangeSelector}
{Whether to show a range selector for the x-axis below the chart.}
{\texttt{true} or \texttt{false}}
{\texttt{false} for in-page chart, \texttt{true} for overlay chart}

\option
{logscale}
{If set, the y-axis uses log scale. Can also be redefined globally (\texttt{pdbfLogscale}).}
{\texttt{true} or \texttt{false}}
{\texttt{false}}

%\option %TODO: not yet implemented
%{overlap}
%{Unused.}
%{\texttt{TODO: ...}}
%{\texttt{TODO: ...}}
%
%\option %TODO: not yet implemented
%{legendpos}
%{Sets the position of the legend of the chart.}
%{\texttt{TODO: ...}}
%{\texttt{TODO: ...}}
%

\noindent\textbf{Note:} To choose JSON-files as your datasource for charts, add to your SQL-FROM clause \texttt{json-files} and to your SQL-WHERE clause \texttt{filename LIKE \%What You Want\%} \\[3pt]

\subsection{Multiplot Chart}
\begin{figure}[h!]%
    Example:\\
    \multiplotChart[width=\textwidth, height=0.5\textwidth, xCount=2, yCount=2, yValues={["b", "b*2"]}, xValues={["a > 2", "a <= 2"]},  topArr={["a smaller or equal 2", "a greater 2"]}, yFirst]{SELECT a,? FROM test WHERE ?;}
    \caption{Multiplot Line Chart}
\end{figure}

\noindent Macros: \\[3pt]
\verb|\multiplotChart[options][queryForOverlay]{queryForPage}| \\
\textbf{Note:} If \verb|queryForOverlay| is omitted the \verb|queryForPage| is used for the overlay\\
\textbf{Note:} Both \verb|queryForOverlay| and \verb|queryForPage| must contain exactly two occurences of the ? character. These are later replaced with values from the xValues/yValues option.\\[8pt]
\noindent Options: \\[3pt]
%
\textbf{Note:} For multiplot charts all options of chart are also valid options.\\[4pt]
%
\option
{xCount}
{Sets number of columns.}
{Number $>$ 0}
{No default value. Value must always be set!}

\option
{yCount}
{Sets number of rows.}
{Number $>$ 0}
{No default value. Value must always be set!}

\option
{leftArr}
{Sets the labels for the left side.}
{Either a JavaScript array of Strings where each string corresponds to exactly one row (e.g \{"row1", "row2"\}) or a JavaScript array of Objects with a c property which corresponds to the number of rows this text should span and a text property which corresponds to the text-string.}
{If this option is not set, then the value of the xunit option is used spanning over the whole site.}

\option
{rightArr}
{Sets the labels for the right side.}
{Either a JavaScript array of Strings where each string corresponds to exactly one row (e.g \{"row1", "row2"\}) or a JavaScript array of Objects with a c property which corresponds to the number of rows this text should span and a text property which corresponds to the text-string.}
{If this option is not set, then the value that is used for the query in this row is used.}

\option
{bottomArr}
{Sets the labels for the bottom side.}
{Either a JavaScript array of Strings where each string corresponds to exactly one column (e.g \{"column1", "column2"\}) or a JavaScript array of Objects with a c property which corresponds to the number of columns this text should span and a text property which corresponds to the text-string.}
{If this option is not set, then the value of the yunit option is used spanning over the whole site.}

\option
{topArr}
{Sets the labels for the top side.}
{Either a JavaScript array of Strings where each string corresponds to exactly one column (e.g \{"column1", "column2"\}) or a JavaScript array of Objects with a c property which corresponds to the number of columns this text should span and a text property which corresponds to the text-string.}
{If this option is not set, then the value that is used for the query in this column is used.}

\option
{xValues}
{The values that replace the first ? character in the query. If yFirst is set, they replace the second ? character in the query instead.}
{A JavaScript array that has as much entries as the xCount options value}
{No default value. Value must always be set!}

\option
{yValues}
{The values that replace the second ? character in the query. If yFirst is set, they replace the first ? character in the query instead.}
{A JavaScript array that has as much entries as the yCount options value}
{No default value. Value must always be set!}

\option
{yFirst}
{If this option is set, then first ? character is replaced with values from yValues option and second ? character is replaced with values from xValues. If this option is not set, then first ? character is replaced with values from xValues option and second ? character is replaced with values from yValues.}
{\texttt{true} or \texttt{false}}
{\texttt{false}}

\option
{forceXequal}
{If this option is set, then all columns have the same x-axis range as the uppermost chart. If this option is not set, then all charts have individual x-axis ranges.}
{\texttt{true} or \texttt{false}}
{\texttt{false}}

\option
{forceYequal}
{If this option is set, then all rows have the same y-axis range as the leftmost chart. If this option is not set, then all charts have individual y-axis ranges.}
{\texttt{true} or \texttt{false}}
{\texttt{false}}

\newpage
\subsection{Pivot Table}
\begin{figure}[h!]%
    Example:\\
    \pivotTable[width=\textwidth, height=0.5\textwidth, rows={["a"]}, aggregationattribute=b]{SELECT * FROM test;}
    \caption{Pivot Table}
\end{figure}

\noindent Macros: \\[3pt]
\verb|\pivotTable[options][queryForOverlay]{queryForPage}| \\
\textbf{Note:} if \verb|queryForOverlay| is omitted the \verb|queryForPage| is used for the overlay\\[8pt]
\noindent Options: \\[3pt]
%
\option
{width}
{Sets the width of the table.}
{\LaTeX~length}
{No default value. Value must always be set!}

\option
{height}
{Sets the height of the table.}
{\LaTeX~length}
{No default value. Value must always be set!}

\option
{quality}
{Sets the quality for the image version of the table in the pdf. 1.0 corresponds roughly to 240 pixels per inch. Can also be redefined globally (\texttt{pdbfQuality}).}
{Number $>$ 0}
{\texttt{1.0}}

\option
{aggregation}
{Sets the aggregation function for the page.}
{\texttt{Count} or \texttt{Count Unique Values} or \texttt{List Unique Values} or \texttt{Sum} or \texttt{Integer Sum} or \texttt{Average} or \texttt{Minimum} or \texttt{Maximum} or \texttt{Sum over Sum} or \texttt{80\% Upper Bound} or \texttt{80\% Lower Bound} or \texttt{Sum as Fraction of Total} or \texttt{Sum as Fraction of Rows} or \texttt{Sum as Fraction of Columns} or \texttt{Count as Fraction of Total} or \texttt{Count as Fraction of Rows} or \texttt{Count as Fraction of Columns}}
{\texttt{Minimum}}

\option
{aggregationBig}
{Sets the aggregation function for the overlay.}
{\texttt{Count} or \texttt{Count Unique Values} or \texttt{List Unique Values} or \texttt{Sum} or \texttt{Integer Sum} or \texttt{Average} or \texttt{Minimum} or \texttt{Maximum} or \texttt{Sum over Sum} or \texttt{80\% Upper Bound} or \texttt{80\% Lower Bound} or \texttt{Sum as Fraction of Total} or \texttt{Sum as Fraction of Rows} or \texttt{Sum as Fraction of Columns} or \texttt{Count as Fraction of Total} or \texttt{Count as Fraction of Rows} or \texttt{Count as Fraction of Columns}}
{If this option is not set, then the value of the aggregation option is used}

\option
{aggregationattribute}
{Sets the attribute for the aggregation function in the page.}
{The name of an attribute that is present in the result of the sql query for the page}
{No default value. Value must always be set!}

\option
{aggregationattributeBig}
{Sets the attribute for the aggregation function in the overlay.}
{The name of an attribute that is present in the result of the sql query for the overlay}
{If this option is not set, then the value of the aggregationattribute option is used}

\option
{cols}
{The attributes for the columns in the pivot table. \textbf{Note:} This feature is currently broken!}
{JavaScript array of strings (e.g. ["a", "b", "c"]). The name of attributes have to be present in the result of the sql query for the overlay and page}
{\texttt{[]} (which means empty array)}

\option
{rows}
{The attributes for the rows in the pivot table.}
{JavaScript array of strings (e.g. ["a", "b", "c"]). The name of attributes have to be present in the result of the sql query for the overlay and page}
{\texttt{[]} (which means empty array)}

\subsection{SQL}
Clickable SQL-statements\\[3pt]
\textbf{Note:} Currently SQL is limited to one line of text. No line breaks are possible.\\
Example:\\[3pt]
\sql{SELECT * FROM test;}\\[3pt]
%
\noindent Macros: \\[3pt]
\verb|\sql[options][textForPage]{queryForOverlay}| \\
\textbf{Note:} if \verb|textForPage| is omitted the \verb|queryForOverlay| is used as text for the page.\\[8pt]

\noindent Options: \\[3pt]

\option
{color}
{The textcolor of the displayed text.}
{\texttt{white}, \texttt{black}, \texttt{red}, \texttt{green}, \texttt{blue}, \texttt{cyan}, \texttt{magenta}, \texttt{yellow} or xcolor syntax (see \href{http://mirror.unicorncloud.org/CTAN/macros/latex/contrib/xcolor/xcolor.pdf}{here}). Can also be redefined globally (\texttt{pdbfDynamicTextColor})}
{\texttt{blue}}

\subsection{DataText}
Write text that is the result of a SQL-statement.\\
\textbf{Note:} Currently DataText is limited to one line of text. No line breaks are possible.\\
\textbf{Note:} DataText can be linked to other PDBF elements (Currently only charts supported). This means that if you move the mouse over this DataText then the specified data in the linked PDBF element is highlighted.\\[3pt]
Example:\\[3pt]
The table ``test'' contains \dataText{SELECT COUNT(*) FROM test;} tuples.\\
The table ``test'' contains the following tuples: \dataText{SELECT * FROM test;}.\\[3pt]
%
\noindent Macros: \\[3pt]
\verb|\dataText[options]{query}| \\[8pt]

\noindent Options: \\[3pt]

\option
{color}
{The textcolor of the displayed text.}
{\texttt{white}, \texttt{black}, \texttt{red}, \texttt{green}, \texttt{blue}, \texttt{cyan}, \texttt{magenta}, \texttt{yellow} or xcolor syntax (see \href{http://mirror.unicorncloud.org/CTAN/macros/latex/contrib/xcolor/xcolor.pdf}{here}). Can also be redefined globally (\texttt{pdbfDynamicTextColor})}
{\texttt{blue}}

\option
{linkTo}
{When this option is set then this DataText and the element with name equal to the value of \texttt{linkTo} are linked. When the cursor is moved above the DataText then in the linked element the specified \texttt{linkSelector} specifies which part of the element is highlighted and annotated with the value of \texttt{linkLabel}.}
{String which value is the name of some other element. \textbf{Currently only chart elements can be used for \texttt{linkTo}.}}
{\texttt{undefined}}

\option
{linkSelector}
{The selector that specifies which part of the \texttt{linkTo} element is highlighted.}
{\texttt{Number}, \texttt{Date} or \texttt{String} depending on the type of the \texttt{linkTo} element. For charts the type depends on the datatype of the xAxis and the value specifies which part of the xAxis is highlighted.}
{\texttt{undefined}}

\option
{linkLabel}
{The text that is displayed near the highlighted area.}
{String}
{\texttt{undefined}}

\subsection{JSON}
Clickable JSON-statement, that will open a seperated view for JSON files. Within this view, other user can input their own JSON files and a highlighted verison, showing differences between the files, will be automatically generated. \\[3pt]
\textbf{Note:} Currently JSON is limited to one line of text. No line breaks are possible.\\
Example:\\[3pt]
You can compare the the cake.json in \textit{examples/comparison/} to this json: \json{cake}\\[3pt]
%
\noindent Macros: \\[3pt]
\verb|\json{filename}| \\

\noindent Options: \\[3pt]

\option
{color}
{The textcolor of the displayed text.}
{\texttt{white}, \texttt{black}, \texttt{red}, \texttt{green}, \texttt{blue}, \texttt{cyan}, \texttt{magenta}, \texttt{yellow} or xcolor syntax (see \href{http://mirror.unicorncloud.org/CTAN/macros/latex/contrib/xcolor/xcolor.pdf}{here}). Can also be redefined globally (\texttt{pdbfDynamicTextColor})}
{\texttt{blue}}

\subsection{DataTable}
Write a \LaTeX\space table that is the result of a SQL-statement.\\[3pt]
Example:\\[3pt]
\dataTable{SELECT * FROM test;}\\[5pt]
%
\noindent Macros: \\[3pt]
\verb|\dataTable[options]{query}| \\[8pt]

\noindent Options: \\[3pt]

\option
{color}
{The textcolor of the displayed text.}
{\texttt{white}, \texttt{black}, \texttt{red}, \texttt{green}, \texttt{blue}, \texttt{cyan}, \texttt{magenta}, \texttt{yellow} or xcolor syntax (see \href{http://mirror.unicorncloud.org/CTAN/macros/latex/contrib/xcolor/xcolor.pdf}{here}). Can also be redefined globally (\texttt{pdbfDynamicTextColor})}
{\texttt{blue}}

\option
{verticalLines}
{Specifies the vertical lines for the table.}
{\texttt{a} for all, \texttt{i} for inner, \texttt{o} for outer, \texttt{n} for none, or you can use the default tabular syntax (e.g. \texttt{|l|c|r|})}
{\texttt{a}}

\option
{horizontalLinesHeader}
{Specifies the horizontal lines for the table header.}
{\texttt{a} for all, \texttt{i} for inner, \texttt{o} for outer, \texttt{n} for none, or you can specify a custom pattern that is repeatedly applied with the following format: h means line, b means no line, the space character delimits format specifiers. E.g. ``hh b'' means: double line followed by no line followed by double line followed by no line ... and so on}
{\texttt{a}}

\option
{horizontalLinesBody}
{Specifies the horizontal lines for the table body, excluding the border above the first tuple, which is controlled by horizontalLinesHeader.}
{\texttt{a} for all, \texttt{i} for inner, \texttt{o} for outer, \texttt{n} for none, or you can specify a custom pattern that is repeatedly applied with the following format: h means line, b means no line, the space character delimits format specifiers. E.g. ``hh b'' means: double line followed by no line followed by double line followed by no line ... and so on}
{\texttt{a}}

\section{F.A.Q.}
\begin{itemize}
\item Q.: The overlay is not on the right position!\\
A.: Most likely you use pages with different sizes in your document or you use the pgfpages package. This is currently not supported by PBDF 
\item Q.: I get \verb|Error: paperwidth value missing! Did you forgot to specify the| \\ \verb|papersize via the geometry package?| when compiling my document.\\
A.: You need to explicitly specify the papersize of your document via the geometry package (e.g. \verb|\usepackage[letterpaper]{geometry}|)
\item Q.: I get \verb|\unhbox \voidb@x \bgroup \let \unhbox \voidb@x \setbox \@tempboxa \hbox| \verb|{u\global \mathchardef \accent@spacefactor \spacefactor }\accent 127 u\egroup| \newline \verb|\spacefactor \accent@spacefactor| or a similar string in a error message.\\
A.: Chances are high that you use the \verb|\usepackage[utf8]{inputenc}| package or a similar package. These are not fully compatible with PDBF. If you really have to use the utf8 package then you cant use special characters inside options and queries of the PDBF package.\\
\item Q.: I get strange errors when running "maven verify".\\
A.: If you are using TexLive 2014 or earlier and get the following errors when running "maven verify":
\begin{lstlisting}[breaklines]
The file is not valid, error(s) :
1.2.1 : Body Syntax error, Single space expected [offset=2786901; key=2786901; line=5 0 obj <<; object=COSObject{5, 0}]
1.2.1 : Body Syntax error, Single space expected [offset=2786635; key=2786635; line=11 0 obj <<; object=COSObject{11, 0}]
1.2.1 : Body Syntax error, EOL expected before the 'endobj' keyword at offset 2786894
1.2.1 : Body Syntax error, Single space expected [offset=2761071; key=2761071; line=3 0 obj <<; object=COSObject{3, 0}]
1.2.1 : Body Syntax error, Single space expected [offset=2761433; key=2761433; line=8 0 obj <<; object=COSObject{8, 0}]
1.2.1 : Body Syntax error, Single space expected [offset=2787764; key=2787764; line=12 0 obj <<; object=COSObject{12, 0}]
\end{lstlisting}
Then you can safely ignore them. They are non critical errors. You can get rid of these errors by upgrading to TexLive 2015 or later.
\end{itemize}
\end{document}
