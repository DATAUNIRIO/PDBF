\documentclass[11pt]{scrartcl}
\usepackage{PDBF}
\usepackage{wrapfig}
\usepackage{hyperref}
\usepackage{ngerman}
\usepackage[a4paper,vmargin={20mm,20mm},hmargin={28mm,28mm}]{geometry}

\title{PDBF Documentation}
\author{
 Patrick Bender\\
  \texttt{s9pabend@stud.uni-saarland.de}
}
\date{\today{}, Saarbr"ucken}

\makeatletter
\def\@maketitle{%
  \vspace*{-\topskip}%
  \begingroup\centering%
  \let \footnote \thanks
  \hrule height \z@%
    {\LARGE \@title \par}%
    \vskip 1.5em 
    {\large
      \lineskip .5em 
      \begin{tabular}[t]{c}%
        \@author
      \end{tabular}\par}%
    \vskip 1em 
    {\large \@date}%
  \par\endgroup%
  \vskip 1.5em%
}
\makeatother

\urlstyle{rm}

\begin{document}
\maketitle

%TODO: Write small abstract 

% TODO: picasso diag

%Required packages:
%\RequirePackage{zref-savepos}
%\RequirePackage{zref-abspage}
%\RequirePackage{zref-user}
%\RequirePackage{xcolor}
%\RequirePackage{graphicx}
%\RequirePackage{xstring}
%\RequirePackage{xparse}
%\RequirePackage{geometry}

%need to use geometry package

%problems with utf8 package. for german umlauts use ngerman package instead. 
%if you really have to use the utf8 package then you cant use special characters inside queries.

\newpage

\section{Chart options}
Macros: \\[3pt]
\verb|\chart[options][queryForOverlay]{queryForPage}| \\
Note: if \verb|queryForOverlay| is omitted the \verb|QueryForPage| is used for the overlay\\[8pt]
\noindent Options: \\[3pt]
\begin{tabular}{ll}
\textbf{chartType} & \\
Desc.:	&	Sets the type of the chart. \\
Argument:	&\verb|line or bar or signatureplot| (more info here)  \\
Default:&	\verb|line| \\[4pt]

\textbf{width} & \\
Desc.:	&	Sets the width of the chart. \\
Argument:	&LaTeX length \\
Default:&	no default value. value must always be set! \\[4pt]

\textbf{height} \\
Desc.:	&	Sets the height of the chart. \\
Argument: &	LaTeX length \\
Default:&	no default value. value must always be set! \\[4pt]

\textbf{xunit} \\
Desc.:	&	Sets the a label for the x axis. \\
Argument: &	String \\
Default:& 	\verb|""| (which means hide) \\[4pt]

\textbf{yunit} \\
Desc.:	&	Sets the a label for the y axis. \\
Argument: 	&String \\
Default:& 	\verb|""| (which means hide) \\[4pt]

\textbf{options} \\
Desc.:	&	Sets options that are directly passed to the c3 chart library (\href{http://c3js.org/reference.html}{$\rightarrow$ Documentation}). \\
& You need to wrap the JSON-String with \verb|{}| if you want to use \verb|[| or \verb|]| or \verb|,|. \\
Argument: 	&JSON-String \\
Default:&	\verb|{}| (which means empty object) \\[4pt]

%TODO: not yet implemented
%\textbf{logscale} \\
%Desc.:	&	If set, the y axis uses log scale. \\
%& Can also be redefined globally (\verb|\pdbfLogscale|) \\
%Argument:&	\verb|true or false| \\
%Default:&	\verb|false| \\[4pt]

%TODO: not yet implemented
%\textbf{overlap} \\
%Desc.:	&	unused \\
%Argument:&	\verb|-| \\
%Default:&	\verb|-| \\[4pt]

legendpos

includeZero

drawPoints

fillGraph

showRangeSelector

\textbf{quality} \\
Desc.:	&	Sets the quality for the image version of the chart in the pdf. 1.0 corresponds roughly to 120 pixels per inch \\
& Can also be redefined globally (\verb|\pdbfQuality|). \\
Argument:	&Number $>$ 0 \\
Default:&	\verb|1.0| \\[4pt]

\textbf{scale} \\
Desc.:	&	Scales the size of the chart. 1.0 corresponds roughly to font-size that is currently choosen in \LaTeX. \\
& Can also be redefined globally (\verb|\pdbfScale|). \\
Argument:	&Number $>$ 0 \\
Default:&	\verb|1.0| \\[4pt]

%TODO: not yet implemented
%\textbf{legendpos} \\
%Desc.: 	&	Sets the position of the legend of the chart. \\
%& Can also be redefined globally (\verb|\pdbfLegendpos|).\\
%Argument:&	\verb|{topright, topcenter, topleft, belowleft, belowcenter, belowright}| \\
%Default:&	\verb|topright| \\[4pt]
\end{tabular} \\

%TODO: Pivot
%TODO: MultiplotChart

%\immediate\write\tempfile{\space\space\space\space\space\space\space\space\space\space\space\space"aggregation": "\pdbf@aggregation",}% 
%		\immediate\write\tempfile{\space\space\space\space\space\space\space\space\space\space\space\space"aggregationattribute": "\pdbf@aggregationattribute",}% 
%		\immediate\write\tempfile{\space\space\space\space\space\space\space\space\space\space\space\space"aggregationBig": "\pdbf@aggregationBig",}% 
%		\immediate\write\tempfile{\space\space\space\space\space\space\space\space\space\space\space\space"aggregationattributeBig": "\pdbf@aggregationattributeBig",}% 
%		\immediate\write\tempfile{\space\space\space\space\space\space\space\space\space\space\space\space"xCount": \pdbf@xCount,}% 
%		\immediate\write\tempfile{\space\space\space\space\space\space\space\space\space\space\space\space"yCount": \pdbf@yCount,}% 
%		\immediate\write\tempfile{\space\space\space\space\space\space\space\space\space\space\space\space"leftArr": "\pdbf@leftArr",}% 
%		\immediate\write\tempfile{\space\space\space\space\space\space\space\space\space\space\space\space"rightArr": "\pdbf@rightArr",}% 
%		\immediate\write\tempfile{\space\space\space\space\space\space\space\space\space\space\space\space"topArr": "\pdbf@topArr",}% 
%		\immediate\write\tempfile{\space\space\space\space\space\space\space\space\space\space\space\space"bottomArr": "\pdbf@bottomArr",}% 
%		\immediate\write\tempfile{\space\space\space\space\space\space\space\space\space\space\space\space"xValues": "\pdbf@xValues",}% 
%		\immediate\write\tempfile{\space\space\space\space\space\space\space\space\space\space\space\space"yValues": "\pdbf@yValues",}% 
%		\immediate\write\tempfile{\space\space\space\space\space\space\space\space\space\space\space\space"yFirst": \pdbf@yFirst,}% 
%		\immediate\write\tempfile{\space\space\space\space\space\space\space\space\space\space\space\space"forceXequal": \pdbf@forceXequal,}% 
%		\immediate\write\tempfile{\space\space\space\space\space\space\space\space\space\space\space\space"forceYequal": \pdbf@forceYequal,}% 
%		\immediate\write\tempfile{\space\space\space\space\space\space\space\space\space\space\space\space"cols": "\pdbf@cols",}%
%		\immediate\write\tempfile{\space\space\space\space\space\space\space\space\space\space\space\space"rows": "\pdbf@rows",}%

%TODO: \sql[textForPage]{queryForOverlay}
%TODO: \dbSQLText{sqlQueryString}
%TODO: \dbSQLFile{fileWithSqlQueries}
%TODO: \dbSQLJDBC{jdbcConnectionURL}{user}{password}{commaSeperatedListOfTableNames}
%jdbcConnectionURL consist of ... (e.g. jdbc:postgresql://localhost:5432/postgres)
%Note: jdbcConnectionURL, user, and password are not stored in the output document

\newpage
\section{F.A.Q.}
\begin{itemize}
\item Q.: The overlay is not on the right position \\
A.: Most likely you use pages with different sizes in your document. This is currently not supported by PBDF 
\item Q.: I get \verb|Error: paperwidth value missing! Did you forgot to specify the| \\ \verb|papersize via the geometry package?| when compiling my document \\
A.: You need to explicitly specify the papersize of your document via the geometry package (e.g. \verb|\usepackage[letterpaper]{geometry}|)
%TODO: check utf8 error message and put a faq entry here
\end{itemize}

\end{document}