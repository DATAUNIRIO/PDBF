\documentclass[11pt]{scrartcl}
\usepackage[utf8]{inputenc}
\usepackage{PDBF}
\usepackage{wrapfig}
\usepackage[hidelinks]{hyperref}
\usepackage[a4paper,vmargin={20mm,40mm},hmargin={20mm,20mm}]{geometry}

\title{PDBF Documentation}
\author{
 Patrick Bender\\
  \texttt{s9pabend@stud.uni-saarland.de}
}
\date{\today{}, Saarbrücken}

\hypersetup{
  colorlinks=true,
  linkcolor=black,
  urlcolor=blue!70!black
}

\makeatletter
\def\@maketitle{%
  \vspace*{-\topskip}%
  \begingroup\centering%
  \let \footnote \thanks
  \hrule height \z@%
    {\LARGE \@title \par}%
    \vskip 1.5em 
    {\large
      \lineskip .5em 
      \begin{tabular}[t]{c}%
        \@author
      \end{tabular}\par}%
    \vskip 1em 
    {\large \@date}%
  \par\endgroup%
  \vskip 1.5em%
}
\makeatother

\urlstyle{rm}

\def\a{5cm}
\def\b{10.5cm}

\def\option#1#2#3#4{%
\noindent \begin{tabular}{|p{\a}|p{\b}|}
\hline
\textbf{#1} & \\
\hline
Desc.: & #2 \\
\hline
Argument: & #3\\
\hline
Default:& #4\\
\hline
\end{tabular} \\[4pt]%
}

\dbSQLText{CREATE TABLE test(a int, b int); INSERT INTO test VALUES(1, 10); INSERT INTO test VALUES(2, 100); INSERT INTO test VALUES(3, 120); INSERT INTO test VALUES(4, 100);}

\begin{document}
\maketitle

\section{Abstract}
PDBF documents are a hybrid format. They are a valid PDF and a valid HTML page at the same time. 
If you change the file extension to PDF and open it with an PDF viewer, you can see the static part of the document. If you change the file extension to HTML and open it with a Browser (currently Chrome/Firefox/Safari/IE 10 supported), you can see the dynamic part of the document. For example, if the static version contains an image of a chart which displays some data, then the dynamic version contains the actual raw data used to render the chart and renders the chart when opening the document. The advantage is that you can open the chart in an overlay view by clicking on the enlarge symbol at the upper left corner and then start to change parameters of the chart. For example you can remove filtering functions that are applied to the raw data or change confidence levels to other values an see the results of these changes directly in the chart. \\

\noindent PDBF files are created from LaTeX source code and a relational database. The raw data can either be a SQL statement string, a file with SQL statements, or contained in a database (currently PostgreSQL/MySQL/MariaDB supported). In the LaTeX code you can then specify how the PDBF element (currently charts/pivot tables/multiplot charts/sql statements are supported) is created from the raw data with options and an SQL query. Read more in the [documentation](http://uds-datalab.github.io/PDBF/) (which is itself a PDBF document).\\

\noindent PDBF toolkit is written in Java and LaTeX and can be used to compile documents on Windows (we are currently working to extend this to Mac and Linux). PDBF documents are also platform independent and run on any desktop OS (Windows, Linux, Mac) with a browser/PDF viewer.\\

\noindent This toolkit is licensed unter the MIT License.\\

\newpage
\noindent Thanks to the authors of:\\
phantomJS (\url{https://github.com/ariya/phantomjs})\\
Apache Commons IO (\url{http://commons.apache.org/proper/commons-io/})\\
Apache Commons Codec (\url{http://commons.apache.org/proper/commons-codec/})\\
google-gson (\url{https://github.com/google/gson})\\
postgreSQL JDBC Driver (\url{https://jdbc.postgresql.org/})\\
MariaDB JDBC Driver (\url{https://mariadb.com/kb/en/mariadb/about-the-mariadb-java-client/})\\
AlaSQL (\url{https://github.com/agershun/alasql})\\
C3 (\url{https://github.com/masayuki0812/c3})\\
D3 (\url{https://github.com/mbostock/d3})\\
Codemirror (\url{https://github.com/codemirror/codemirror})\\
google-diff-match-patch (\url{https://code.google.com/p/google-diff-match-patch/})\\
explorercanvas (\url{https://code.google.com/p/explorercanvas/})\\
DataTables (\url{https://github.com/DataTables/DataTables})\\
jQuery (\url{https://github.com/jquery/jquery})\\
jQuery-UI (\url{https://github.com/jquery/jquery-ui})\\
jStat (\url{https://github.com/jstat/jstat})\\
pivottable (\url{https://github.com/nicolaskruchten/pivottable})\\
PDF.js (\url{https://github.com/mozilla/pdf.js})\\
google-closure-compiler (\url{https://github.com/google/closure-compiler})\\
\newpage
\tableofcontents
\newpage

\section{Requirements}
These \LaTeX\relax packages are required for PDBF documents:\\
zref\\
xcolor\\
graphicx\\
xstring\\
xparse\\
geometry\\

\noindent Furthermore it is needed that your \LaTeX\relax document uses the geometry package to specify the page size.\\

\noindent \textbf{Warning:} There are problems with inputenc UTF8 package. If you really have to use the inputenc UTF8 package then you cant use non-ASCII characters inside queries.

\section{SQL queries for PDBF elements}
PDBF uses AlaSQL as database engine. For all available features visit \url{https://github.com/agershun/alasql}. Some functionality might not yet be available because we currently use not the latest version.\\
Mostly sql syntax is standart sql, but for example attributes with spaces in their name have to be sourrounded by square brackets. E.g. SELECT a AS [This is a test] FROM test;\\

\noindent We extended AlaSQL with the following statistical sql functions:\\[5pt]
\verb|GRUBBS_FILTER(arr, alpha)|: \\
\indent Desc.: Filters an array of values with a \href{http://en.wikipedia.org/wiki/Grubbs'_test_for_outliers}{grubbs test}\\
\indent Returns the filtered array.\\[3pt]
\indent \option
{arr}
{Array on which the operation is performed.}
{JavaScript array}
{No default value. Value must always be set!}
\indent \option
{alpha}
{significance level}
{Number $>$ 0 AND Number $<$ 1}
{\texttt{0.05}}\\[3pt]
%
\verb|MEAN(arr)|: \\
\indent Desc.: Calculates the \href{http://en.wikipedia.org/wiki/Arithmetic_mean}{arithmetic mean} of an array of values.\\
\indent Returns the arithmetic mean of the array.\\[3pt]
\indent \option
{arr}
{Array on which the operation is performed.}
{JavaScript array}
{No default value. Value must always be set!}\\[3pt]
%
\verb|STDDEV_SAMP(arr)|: \\
\indent Desc.: Calculates the \href{http://en.wikipedia.org/wiki/Standard_deviation}{ sample standard deviation} of an array of values.\\
\indent Returns the sample standard deviation of the array.\\[3pt]
\indent \option
{arr}
{Array on which the operation is performed.}
{JavaScript array}
{No default value. Value must always be set!}\\[3pt]
%
\verb|MARGIN_OF_ERROR(arr, alpha)|: \\
\indent Desc.: Calculates the \href{http://en.wikipedia.org/wiki/Margin_of_error}{margin of error} of an array of values.\\
\indent Returns the margin of error of the array.\\[3pt]
\indent \option
{arr}
{Array on which the operation is performed.}
{JavaScript array}
{No default value. Value must always be set!}
\indent \option
{alpha}
{significance level}
{Number $>$ 0 AND Number $<$ 1}
{\texttt{0.05}}\\[3pt]
%
\verb|CONF_INT(arr, alpha)|: \\
\indent Desc.: Calculates the \href{http://en.wikipedia.org/wiki/Confidence_interval}{confidence interval} of an array.\\
\indent Returns the confidence interval of the array.\\[3pt]
\indent \option
{arr}
{Array on which the operation is performed.}
{JavaScript array}
{No default value. Value must always be set!}
\indent \option
{alpha}
{significance level}
{Number $>$ 0 AND Number $<$ 1}
{\texttt{0.05}}\\[3pt]
%
\verb|T_TEST(arr1, arr2, alpha)|: \\
\indent Desc.: Performs a \href{http://en.wikipedia.org/wiki/Student's_t-test}{Student's t-test} on two arrays.\\
\indent Returns true if they are indistinguishable, and returns false otherwise.\\[3pt]
\indent \option
{arr1, arr2}
{Arrays on which the operation is performed.}
{JavaScript array}
{No default value. Value must always be set!}
\indent \option
{alpha}
{significance level}
{Number $>$ 0 AND Number $<$ 1}
{\texttt{0.05}}\\[3pt]
%
\verb|WELCH_TEST(arr1, arr2, alpha)|:\\
\indent Desc.: Performs a \href{http://en.wikipedia.org/wiki/Welch's_t_test}{Welch's t-test} on two arrays.\\
\indent Returns true if they are indistinguishable, and returns false otherwise.\\[3pt]
\indent \option
{arr1, arr2}
{Arrays on which the operation is performed.}
{JavaScript array}
{No default value. Value must always be set!}
\indent \option
{alpha}
{significance level}
{Number $>$ 0 AND Number $<$ 1}
{\texttt{0.05}}\\[3pt]
%
\section{Sql data sources}
\noindent Macros: \\[3pt]
\verb|\dbSQLText{sqlQueryString}| \\
\verb|\dbSQLFile{fileWithSqlQueries}| \\
\verb|\dbSQLJDBC{jdbcConnectionURL}{user}{password}{commaSeperatedListOfTableNames}| \\
\textbf{Note:} jdbcConnectionURL consist of jdbc followed by the name of the dbms (currently only postgresql and mysql are supported) followed by the url of the dbms followed by the database name (e.g. jdbc:postgresql://localhost:5432/postgres).\\
\textbf{Note:} jdbcConnectionURL, user, and password are not stored in the output document.

\newpage
\section{PDBF elements}
Currently there are four different PDBF elements: charts, pivot tables, multiplot charts, and sql statements.

\subsection{Chart}
\begin{figure}[h!]%
\hspace{-218pt}Examples:\\
    \centering
    \begin{minipage}{.48\textwidth}
    \chart[width=\textwidth, height=0.6\textwidth, chartType=line]{SELECT * FROM test;}
    \caption{Line Chart}
    \end{minipage}
    \hspace{11pt}
    \begin{minipage}{.48\textwidth}
    \chart[width=\textwidth, height=0.6\textwidth, chartType=bar]{SELECT * FROM test;}
    \caption{Bar Chart}
    \end{minipage} \\[8pt]
    
    \begin{minipage}{1.0\textwidth}
    \chart[width=\textwidth, height=0.36\textwidth, chartType=signatureplot]{SELECT b FROM test;}
    \caption{Signatureplot Chart}
    \end{minipage}
%    \begin{minipage}{.48\textwidth}
%    \chart[width=\textwidth, height=0.8\textwidth, chartType=signatureplot]{SELECT * FROM test;}
%    \caption{Signatureplot Chart}
%    \end{minipage}
%    \hspace{11pt}
%    \begin{minipage}{.48\textwidth}
%    \chart[width=\textwidth, height=0.8\textwidth, chartType=Line, xunit=Date, yunit={Runtime [in sec]}, options={"strokeWidth": 0, "drawPoints": true, "pointSize": 3, "highlightCircleSize": 4, "xRangePad": 10}]{SELECT date, runtimeA AS [engine A], runtimeB AS [engine B] FROM data2;}
%    \caption{Multi-column Dot Chart}
%    \end{minipage}
\end{figure}

\noindent Macros: \\[3pt]
\verb|\chart[options][queryForOverlay]{queryForPage}| \\
\textbf{Note:} if \verb|queryForOverlay| is omitted the \verb|queryForPage| is used for the overlay.\\
\textbf{Note:} The first column of the query result is used for the x-Axis, all other columns are used for y-Axis.\\[8pt] 
\noindent Options:\\[3pt]
%
Common options\\[4pt]
%
\option
{width}
{Sets the width of the chart.}
{\LaTeX~length}
{No default value. Value must always be set!}

\option
{height}
{Sets the height of the chart.}
{\LaTeX~length}
{No default value. Value must always be set!}

\option
{quality}
{Sets the quality for the image version of the chart in the pdf. 1.0 corresponds roughly to 120 pixels per inch. Can also be redefined globally (\texttt{pdbfQuality}).}
{Number $>$ 0}
{\texttt{1.0}}

\option
{scale}
{Scales the size of the chart. 1.0 corresponds roughly to font-size that is currently choosen in \LaTeX. Can also be redefined globally (\texttt{pdbfScale}).}
{Number $>$ 0}
{\texttt{1.0}}

\option
{chartType}
{Sets the type of the chart.}
{\texttt{line} or \texttt{bar} or \texttt{signatureplot}}
{\texttt{line}}

\option
{xunit}
{Sets the a label for the x-axis.}
{String}
{\texttt{""} (which means hide)}

\option
{yunit}
{Sets the a label for the y-axis.}
{String}
{\texttt{""} (which means hide)}

\option
{options}
{Sets options that are directly passed to the c3 chart library (\href{http://c3js.org/reference.html}{$\rightarrow$ Documentation}). You need to wrap the JSON-String with \texttt{\{\}} if you want to use \texttt{[} or \texttt{]} or \texttt{,}.}
{JSON-String}
{\texttt{\{\}} (which means empty object)}

\option %TODO: not yet implemented
{logscale}
{If set, the y axis uses log scale. Can also be redefined globally (\texttt{pdbfLogscale}).}
{\texttt{true} or \texttt{false}}
{\texttt{false}}

\option %TODO: not yet implemented
{overlap}
{Unused.}
{\texttt{TODO: ...}}
{\texttt{TODO: ...}}

\option %TODO: not yet implemented
{legendpos}
{Sets the position of the legend of the chart.}
{\texttt{TODO: ...}}
{\texttt{TODO: ...}}

\option %TODO: not yet implemented
{includeZero}
{Set the minimum of the range of the y-axis to zero.}
{\texttt{true} or \texttt{false}}
{\texttt{false}}

\option %TODO: not yet implemented
{drawPoints}
{Whether to show each point in line.}
{\texttt{true} or \texttt{false}}
{\texttt{false}}

\option %TODO: not yet implemented
{fillGraph}
{Whether to fill the area below the graph.}
{\texttt{true} or \texttt{false}}
{\texttt{false}}

\option %TODO: not yet implemented
{showRangeSelector}
{Whether to show a range selector for the x-axis below the chart.}
{\texttt{true} or \texttt{false}}
{\texttt{false}}

\newpage
\subsection{Multiplot Chart}
\begin{figure}[h!]%
    Example:\\
    \multiplotChart[width=\textwidth, height=0.5\textwidth, xCount=2, yCount=2, yValues={["b", "b*2"]}, xValues={["a > 2", "a <= 2"]}, yFirst]{SELECT a,? FROM test WHERE ?;}
    \caption{Multiplot Line Chart}
\end{figure}

\noindent Macros: \\[3pt]
\verb|\multiplotChart[options][queryForOverlay]{queryForPage}| \\
\textbf{Note:} If \verb|queryForOverlay| is omitted the \verb|queryForPage| is used for the overlay\\
\textbf{Note:} Both \verb|queryForOverlay| and \verb|queryForPage| must contain exactly two occurences of the ? character. These are later replaced with values from the xValues/yValues option.\\[8pt]
\noindent Options: \\[3pt]
%
\textbf{Note:} For multiplot charts all options of chart are also valid options.\\[4pt]
%
\option
{xCount}
{Sets number of columns.}
{Number $>$ 0}
{No default value. Value must always be set!}

\option
{yCount}
{Sets number of rows.}
{Number $>$ 0}
{No default value. Value must always be set!}

\option
{leftArr}
{Sets the labels for the left side.}
{Either a JavaScript array of Strings where each string corresponds to exactly one row (e.g \{"row1", "row2"\}) or a JavaScript array of Objects with a c property which corresponds to the number of rows this text should span and a text property which corresponds to the text-string.}
{If this option is not set, then the value of the xunit option is used spanning over the whole site.}

\option
{rightArr}
{Sets the labels for the right side.}
{Either a JavaScript array of Strings where each string corresponds to exactly one row (e.g \{"row1", "row2"\}) or a JavaScript array of Objects with a c property which corresponds to the number of rows this text should span and a text property which corresponds to the text-string.}
{If this option is not set, then the value that is used for the query in this row is used.}

\option
{bottomArr}
{Sets the labels for the bottom side.}
{Either a JavaScript array of Strings where each string corresponds to exactly one column (e.g \{"column1", "column2"\}) or a JavaScript array of Objects with a c property which corresponds to the number of columns this text should span and a text property which corresponds to the text-string.}
{If this option is not set, then the value of the yunit option is used spanning over the whole site.}

\option
{topArr}
{Sets the labels for the tod side.}
{Either a JavaScript array of Strings where each string corresponds to exactly one column (e.g \{"column1", "column2"\}) or a JavaScript array of Objects with a c property which corresponds to the number of columns this text should span and a text property which corresponds to the text-string.}
{If this option is not set, then the value that is used for the query in this column is used.}

\option
{xValues}
{The values that replace the first ? character in the query. If yFirst is set, they replace the second ? character in the query instead.}
{A JavaScript array that has as much entries as the xCount options value}
{No default value. Value must always be set!}

\option
{yValues}
{The values that replace the second ? character in the query. If yFirst is set, they replace the first ? character in the query instead.}
{A JavaScript array that has as much entries as the yCount options value}
{No default value. Value must always be set!}

\option
{yFirst}
{If this option is set, then first ? character is replaced with values from yValues option and second ? character is replaced with values from xValues. If this option is not set, then first ? character is replaced with values from xValues option and second ? character is replaced with values from yValues.}
{\texttt{true} or \texttt{false}}
{\texttt{false}}

\option
{forceXequal}
{If this option is set, then all columns have the same x-axis range as the uppermost chart. If this option is not set, then all charts have individual x-axis ranges.}
{\texttt{true} or \texttt{false}}
{\texttt{false}}

\option
{forceYequal}
{If this option is set, then all rows have the same y-axis range as the leftmost chart. If this option is not set, then all charts have individual y-axis ranges.}
{\texttt{true} or \texttt{false}}
{\texttt{false}}

\newpage
\subsection{Pivot Table}
\begin{figure}[h!]%
    Example:\\
    \pivotTable[width=\textwidth, height=0.5\textwidth, rows={["a"]}, aggregationattribute=b]{SELECT * FROM test;}
    \caption{Pivot Table}
\end{figure}

\noindent Macros: \\[3pt]
\verb|\pivotTable[options][queryForOverlay]{queryForPage}| \\
\textbf{Note:} if \verb|queryForOverlay| is omitted the \verb|queryForPage| is used for the overlay\\[8pt]
\noindent Options: \\[3pt]
%
\option
{width}
{Sets the width of the chart.}
{\LaTeX~length}
{No default value. Value must always be set!}

\option
{height}
{Sets the height of the chart.}
{\LaTeX~length}
{No default value. Value must always be set!}

\option
{quality}
{Sets the quality for the image version of the chart in the pdf. 1.0 corresponds roughly to 120 pixels per inch. Can also be redefined globally (\texttt{pdbfQuality}).}
{Number $>$ 0}
{\texttt{1.0}}

\option
{scale}
{Scales the size of the chart. 1.0 corresponds roughly to font-size that is currently choosen in \LaTeX. Can also be redefined globally (\texttt{pdbfScale}).}
{Number $>$ 0}
{\texttt{1.0}}

\option
{aggregation}
{Sets the aggregation function for the page.}
{\texttt{Count} or \texttt{Count Unique Values} or \texttt{List Unique Values} or \texttt{Sum} or \texttt{Integer Sum} or \texttt{Average} or \texttt{Minimum} or \texttt{Maximum} or \texttt{Sum over Sum} or \texttt{80\% Upper Bound} or \texttt{80\% Lower Bound} or \texttt{Sum as Fraction of Total} or \texttt{Sum as Fraction of Rows} or \texttt{Sum as Fraction of Columns} or \texttt{Count as Fraction of Total} or \texttt{Count as Fraction of Rows} or \texttt{Count as Fraction of Columns}}
{\texttt{Minimum}}

\option
{aggregationBig}
{Sets the aggregation function for the overlay.}
{\texttt{Count} or \texttt{Count Unique Values} or \texttt{List Unique Values} or \texttt{Sum} or \texttt{Integer Sum} or \texttt{Average} or \texttt{Minimum} or \texttt{Maximum} or \texttt{Sum over Sum} or \texttt{80\% Upper Bound} or \texttt{80\% Lower Bound} or \texttt{Sum as Fraction of Total} or \texttt{Sum as Fraction of Rows} or \texttt{Sum as Fraction of Columns} or \texttt{Count as Fraction of Total} or \texttt{Count as Fraction of Rows} or \texttt{Count as Fraction of Columns}}
{If this option is not set, then the value of the aggregation option is used}

\option
{aggregationattribute}
{Sets the attribute for the aggregation function in the page.}
{The name of an attribute that is present in the result of the sql query for the page}
{No default value. Value must always be set!}

\option
{aggregationattributeBig}
{Sets the attribute for the aggregation function in the overlay.}
{The name of an attribute that is present in the result of the sql query for the overlay}
{If this option is not set, then the value of the aggregationattribute option is used}

\option
{cols}
{The attributes for the columns in the pivot table.}
{JavaScript array of strings (e.g. ["a", "b", "c"]). The name of attributes have to be present in the result of the sql query for the overlay and page}
{\texttt{[]} (which means empty array)}

\option
{rows}
{The attributes for the rows in the pivot table.}
{JavaScript array of strings (e.g. ["a", "b", "c"]). The name of attributes have to be present in the result of the sql query for the overlay and page}
{\texttt{[]} (which means empty array)}

\newpage
\subsection{Sql}
Example:\\[3pt]
\sql{SELECT * FROM test;}\\[3pt]
%
\noindent Macros: \\[3pt]
\verb|\sql[textForPage]{queryForOverlay}| \\
\textbf{Note:} if \verb|textForPage| is omitted the \verb|queryForOverlay| is used as text for the page.\\[8pt]

\section{F.A.Q.}
\begin{itemize}
\item Q.: The overlay is not on the right position.\\
A.: Most likely you use pages with different sizes in your document. This is currently not supported by PBDF 
\item Q.: I get \verb|Error: paperwidth value missing! Did you forgot to specify the| \\ \verb|papersize via the geometry package?| when compiling my document.\\
A.: You need to explicitly specify the papersize of your document via the geometry package (e.g. \verb|\usepackage[letterpaper]{geometry}|)
\item Q.: I get \verb|\unhbox \voidb@x \bgroup \let \unhbox \voidb@x \setbox \@tempboxa \hbox| \verb|{u\global \mathchardef \accent@spacefactor \spacefactor }\accent 127 u\egroup| \newline \verb|\spacefactor \accent@spacefactor| or a similar string in a error message.\\
A.: Chances are high that you use the \verb|\usepackage[utf8]{inputenc}| package or a similar package. These are not fully compatible with PDBF. If you really have to use the utf8 package then you cant use special characters inside PDBF commands.
\end{itemize}

\end{document}
