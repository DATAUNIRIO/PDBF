\documentclass[10pt]{article}
\usepackage{pdbf}
\usepackage[a4paper]{geometry}

\dbSQLText{CREATE TABLE test(a int, b int, [attribute with space] int); INSERT INTO test VALUES(1, 1, 1); INSERT INTO test VALUES(2, 2, 4); INSERT INTO test VALUES(3, 3, 8); INSERT INTO test VALUES(4, 4, 16);}
%\dbSQLFile{somedb.sql}
%\dbSQLJDBC{jdbc:postgresql://localhost:5432/postgres}{postgres}{test}{testtable}%
%\dbCSVFile{result.csv}{test}

\jsonDirectory{examples/json}
\jsonFile{examples/json/nested/cake.json}{cake}

\begin{document}
\noindent

\sql{SELECT * FROM test;}\\[3pt]
The table ``test'' contains \dataText{SELECT COUNT(*) FROM test;} tuples.\\[3pt]

The following points are linked to the chart on the left side below: \dataText[linkTo=testname, linkSelector=2, linkLabel=Value at 2]{SELECT * FROM test WHERE a=2;}, \dataText[linkTo=testname, linkSelector=3, linkLabel=Value at 3]{SELECT * FROM test WHERE a=3;}.\\[3pt]
\chart[width=0.45\textwidth, height=0.3\textwidth, logscale=true, chartType=line, name=testname]{SELECT a, b, [attribute with space] FROM test;}
\chart[width=0.45\textwidth, height=0.3\textwidth, chartType=bar]{SELECT a, b, [attribute with space] FROM test;}\\[3pt]

{\fontfamily{phv} \tiny \selectfont
	\noindent
	Change the font before this line to Helvetica and fontsize to tiny.\\[3pt]
	\chart[width=0.6\textwidth, height=0.25\textwidth, chartType=compareToBest]{SELECT b FROM test;}\\[3pt]
	\multiplotChart[width=0.6\textwidth, height=0.5\textwidth, xCount=2, yCount=2, yValues={["b", "b*2"]}, xValues={["a <= 2", "a > 2"]}, topArr={["a smaller or equal 2", "a greater 2"]}, yFirst]{SELECT a,? FROM test WHERE ?;}\\[3pt]
	\pivotTable[width=0.10\textwidth, height=0.15\textwidth, rows={["a"]}, aggregationattribute=b]{SELECT * FROM test;}
	\dataTable[verticalLines=i, horizontalLinesHeader=b hh, horizontalLinesBody=i]{SELECT * FROM test;}\\[3pt]
}

The following Chart uses information from JSON-files. \\[3pt]
\chart[width=0.45\textwidth, height=0.3\textwidth, logscale=true, chartType=line]{SELECT a, b, c FROM json_files WHERE json_filename LIKE 'run\%';}
\chart[width=0.45\textwidth, height=0.3\textwidth, logscale=true, chartType=line, name=testname]{SELECT a, b, c FROM json_files WHERE c > 5 AND json_filename LIKE 'run\%';} \\[3pt]

To check if the results from JSON-files contain correct data, one can insert his own data for comparision. Examples to drag and drop can be found in the examples folder.
\begin{itemize}
\item This JSON has correct values \json{run1}
\item Here some values are changed \json{run4}
\item And here are many nested values \json{cake}
\end{itemize}


\attachment[click me][view][local]{examples/attachment/pdf\_reference.pdf} to view pdf\_reference.pdf embedded from local storage. \\
\attachment[click me][download][local]{examples/attachment/pdf\_reference.pdf} to download pdf\_reference.pdf embedded from local storage. \\
\attachment[click me][view][local]{examples/attachment/big\_mountain.png} to view big\_mountain.png embedded from local storage. \\
\attachment[click me][view][web]{https://www.google.de/} to view Google.html embedded from web storage. \\
\attachment[click me][download][local]{examples/attachment/Google.html} to download Google.html embedded from local storage. \\
\attachment[click me][view][web]{http://www.khazanah.com.my/khazanah/files/20/200f21f3-07ff-4903-ab99-7c0cb557eb51.pdf} to view pdf\_reference.pdf embedded from web storage. \\
\attachment[click me][download][web]{http://www.khazanah.com.my/khazanah/files/20/200f21f3-07ff-4903-ab99-7c0cb557eb51.pdf} to download pdf\_reference.pdf embedded from web storage. \\

\graph[width=0.75\textwidth, height=0.5\textwidth]{examples/graphs/graph.txt} \\

\end{document}