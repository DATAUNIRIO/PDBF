\documentclass[10pt]{beamer}
\usetheme{Frankfurt}
\usepackage{PDBF}
\setbeameroption{hide notes}

\addtobeamertemplate{note page}{\setlength{\parskip}{12pt}}
\makeatletter
\beamertemplatenavigationsymbolsempty
\setbeamerfont{footline}{size=\normalsize}
\setbeamertemplate{footline}{{\par\vspace*{-2cm}\hspace*{\fill}\vspace*{0.25cm}\insertframenumber~~~}}
\setbeamertemplate{items}[default]
\setbeamertemplate{sections/subsections in toc}[sections numbered]
\setbeamertemplate{subsection in toc}{}
\setbeamercolor{title}{fg=white,bg=red!50!green!70}
\setbeamercolor{frametitle}{fg=white,bg=red!50!green!70}
\setbeamercolor{subsection in toc}{fg=black}
\setbeamercolor{section in toc}{fg=black}
\setbeamercolor{item}{fg=black}
\setbeamertemplate{caption}[numbered]
\setbeamercolor{caption name}{fg=black}
\def\beamer@framenotesbegin{% at beginning of slide
  \gdef\beamer@noteitems{}%
  \gdef\beamer@notes{{}}% used to be totally empty.
}
\addtobeamertemplate{frametitle}{}{\vspace{0.4cm}}
\makeatother

\title{PDBF}   
\author{
 Patrick Bender\\
  \texttt{s9pabend@stud.uni-saarland.de}
}
\date{}

\renewcommand\pdbfScale{0.833}

\dbSQLText{CREATE TABLE table1 (column1 INTEGER, column2 DATE, column3 FLOAT);}
\dbSQLFile{somedb.sql}
\dbSQLJDBC{jdbc:postgresql://localhost:5432/postgres}{postgres}{test}{testtest, dim_time, dim_compiler, dim_dataset, dim_implementation, dim_layout, dim_query, dim_machine, fact_experiment}%

\begin{document}

\begin{frame}
\vspace{-1.0cm}
\titlepage
\vspace{-2.5cm}
\begin{center}
\includegraphics[scale=0.25]{C:/Users/User/Downloads/UdS-Logo.jpg}
\end{center}
Presenter: Patrick Bender
\end{frame}

\section{Goals} 
\subsection*{}
\vspace{1.0cm}
\begin{frame}
	\begin{Large}\begin{center}What are the goals?\end{center}\end{Large}
\end{frame}

\section{SQL Queries} 
\subsection*{}
\begin{frame}\frametitle{Simple Queries} 
	\sql{SELECT * FROM test;}\\[12pt]
	\sql{SELECT col2 FROM test WHERE col1=1;}\\[12pt]
	\sqlmult{INSERT INTO test}{INSERT INTO test VALUES (1, 123);}\\
	\sqlmult{VALUES (1, 123);}{INSERT INTO test VALUES (1, 123);}
\end{frame}

\subsection*{}
\begin{frame}\frametitle{Advanced Queries}
\vspace{-0.4cm}
	\def\longquery{SELECT
m.cpu + ' (' +  m.cpuarchitecture + ')' AS Machine,
d.tableschema   AS [Table schema],
c.compiler      AS Compiler,
c.optlevel      AS [Opt. level],
COALESCE(cast(l.chunksize AS VARCHAR), 'n/a') AS [Chunk size (tuples)],
IIF(i.templateparam, 'Compile time', 'Runtime') AS [Chunk size provided at],
IIF(i.logscaleparam, 'Powers of 2', 'Multiples of 2') AS [Chunk sizes considered],
t.timestamp 	AS [Time of experiment],
ROUND(v.runtime, 3) 		AS [Query time (in Sec)]
FROM fact_experiment AS v
    JOIN dim_machine m ON m.id = v.machineid
    JOIN dim_dataset d ON d.id = v.datasetid
    JOIN dim_compiler c ON c.id = v.compilerid
    JOIN dim_layout l ON l.id = v.layoutid
    JOIN dim_implementation i ON i.id = v.implementationid
    JOIN dim_query q ON q.id = v.queryid
    JOIN dim_time t ON t.id = v.timeid
ORDER BY m.id DESC, d.tableschema, q.querytext,
    c.compiler, c.optlevel, i.templateparam, i.logscaleparam, l.chunksize
LIMIT 100;}
	\begin{scriptsize}
	\sqlmult{SELECT}{\longquery}\\
	\sqlmult{m.cpu + ' (' +  m.cpuarchitecture + ')' AS Machine,}{\longquery}\\
	\sqlmult{d.tableschema   AS [Table schema],}{\longquery}\\
	\sqlmult{c.compiler      AS Compiler,}{\longquery}\\
	\sqlmult{c.optlevel      AS [Opt. level],}{\longquery}\\
	\sqlmult{COALESCE(cast(l.chunksize AS VARCHAR), 'n/a') AS [Chunk size (tuples)],}{\longquery}\\
	\sqlmult{IIF(i.templateparam, 'Compile time', 'Runtime') AS [Chunk size provided at],}{\longquery}\\
	\sqlmult{IIF(i.logscaleparam, 'Powers of 2', 'Multiples of 2') AS [Chunk sizes considered],}{\longquery}\\
	\sqlmult{t.timestamp 	AS [Time of experiment],}{\longquery}\\
	\sqlmult{ROUND(v.runtime, 3) 		AS [Query time (in Sec)]}{\longquery}\\
	\sqlmult{FROM fact\_experiment AS v}{\longquery}\\
	\sqlmult{    JOIN dim\_machine m ON m.id = v.machineid}{\longquery}\\
	\sqlmult{    JOIN dim\_dataset d ON d.id = v.datasetid}{\longquery}\\
	\sqlmult{    JOIN dim\_compiler c ON c.id = v.compilerid}{\longquery}\\
	\sqlmult{    JOIN dim\_layout l ON l.id = v.layoutid}{\longquery}\\
	\sqlmult{    JOIN dim\_implementation i ON i.id = v.implementationid}{\longquery}\\
	\sqlmult{    JOIN dim\_query q ON q.id = v.queryid}{\longquery}\\
	\sqlmult{    JOIN dim\_time t ON t.id = v.timeid}{\longquery}\\
	\sqlmult{ORDER BY m.id DESC, d.tableschema, q.querytext,}{\longquery}\\
	\sqlmult{    c.compiler, c.optlevel, i.templateparam, i.logscaleparam, l.chunksize}{\longquery}\\
	\sqlmult{LIMIT 100;}{\longquery}\\
	\end{scriptsize}
\end{frame}

\section{Charts} 
\subsection*{}
\begin{frame}\frametitle{Line Chart with little data}
	{ \centering 
	\chart[width=0.8\textwidth, height=0.6\textwidth, chartType=Line, xunit=Money (in €), yunit=Date, options={"title": "Title"}]{SELECT * FROM data1;}%
	\\}
\end{frame}

\subsection*{}
\begin{frame}\frametitle{Line Chart with much data}
	{\centering
	    \chart[width=1.0\textwidth, height=0.5\textwidth, chartType=Line, xunit=Row Number, yunit={Runtime [in sec]}]{SELECT ROWNUM(), runtime FROM fact_experiment;}%
	\\}
\end{frame}

\subsection*{}
\begin{frame}\frametitle{Other Chart types}
	\begin{figure}[h]
    \centering
    \begin{minipage}{.48\textwidth}
    \chart[width=\textwidth, height=0.8\textwidth, chartType=Line, xunit=Date, yunit={Runtime [in sec]}]{SELECT date, runtimeA AS engine_A, runtimeB AS engine_B FROM data2;}%
    \caption{Multi-column Line Chart}
    \end{minipage}
    \hspace{5pt}
    \begin{minipage}{.48\textwidth}
    \chart[width=\textwidth, height=0.8\textwidth, chartType=Line, xunit=Date, yunit={Runtime [in sec]}]{SELECT date, runtimeA AS [engine A], runtimeB AS [engine B] FROM data2;}%
    \caption{Multi-column Bar Chart}
    \end{minipage}\\[8pt]
    \end{figure}
\end{frame}

\subsection*{}
\begin{frame}\frametitle{Other Chart types}
	\begin{figure}[h]
    \centering
    \begin{minipage}{.48\textwidth}
    \chart[width=\textwidth, height=0.8\textwidth, chartType=Bar, xunit=Date, yunit={Runtime [in sec]}, overlap=2]{SELECT date, runtimeA AS [engine A], runtimeB AS [engine B] FROM data2;}
    \caption{Multi-column overlapping Bar Chart}
    \end{minipage}
    \hspace{5pt}
    \begin{minipage}{.48\textwidth}
    \chart[width=\textwidth, height=0.8\textwidth, chartType=Line, xunit=Date, yunit={Runtime [in sec]}, options={"strokeWidth": 0, "drawPoints": true, "pointSize": 3, "highlightCircleSize": 4, "xRangePad": 10}]{SELECT date, runtimeA AS [engine A], runtimeB AS [engine B] FROM data2;}
    \caption{Multi-column Dot Chart}
    \end{minipage}
    \end{figure}
\end{frame}

\section*{}
\begin{frame}
\vspace{1.0cm}
	\begin{Large}\begin{center}Thank you for your attention!\end{center}\end{Large}
\end{frame}

\begin{frame}
\vspace{1.0cm}
	\begin{Large}\begin{center}Questions?\end{center}\end{Large}
\end{frame}

\end{document}

